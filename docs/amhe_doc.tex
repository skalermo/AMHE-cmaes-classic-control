\documentclass[12pt,a4paper]{article}

% for polish language
\usepackage{polski}

% for some math symbols
\usepackage{amssymb}

% correct footnotes placement
\usepackage[bottom]{footmisc}

% for \say command
\usepackage{dirtytalk}

% change title of the bibliography
\def\bibname{Referencje}\let\refname\bibname

% for advanced math formulas
\usepackage{mathtools}

\usepackage{listings}

% links
\usepackage{hyperref}
% \hypersetup{
%     colorlinks,
%     citecolor=black,
%     filecolor=black,
%     linkcolor=black,
%     urlcolor=blue
% }

% image displaying
\usepackage{subcaption}
\usepackage{graphicx}

\usepackage{multirow} 
\usepackage{makecell}

\title{Dokumentacja do projektu z AMHE}

\author{Roman Moskalenko \and Pavel Tarashkevich}
\date{}

\begin{document}

\maketitle

\section{Treść zadania}

\textbf{Zadanie 10.}\\

Zaprojektuj algorytm ewolucyjny i zbadaj jego działanie w jednym z zadań zdefiniowanych w ramach środowiska OpenAI poświęconym kontroli.

Specyfikacja zadań znajduje się pod poniższym linkiem:

\href{https://gym.openai.com/envs/#classic\_control}{gym.openai.com/envs/\#classic\_control}

Porównaj wyniki swojego rozwiązania z dwoma wybranymi rozwiązaniami, które bazują na
innych podejściach.


\section{Projekt wstępny}

\subsection{Opis problemu i jego sposób rozwiązania}

Zadania kontroli klasycznej zakładają sterowanie obiektu fizycznego w sposób, 
jak najbardziej efektywny dla jego stanu. Funkcja sterowania może być bardzo
skomplikowana, uciążliwe jest jej zaprojektowanie w sposób ręczny. 

Algorytmy uczenia się ze wzmocnieniem (dalej RL) są wykorzystywane w zadaniach związanych z kontrolą.
Do ewaluacji tych algorytmów powstał zestaw środowisk OpenAI gym.

Alternatywnym podejściem do zadań kontroli jest bezpośrednie stosowanie sieci 
neuronowych, m.i. podejście neuroewolucji zastosowane w \cite{scalable_alternative}, 
które zakłada połączenie sieci neuronowych i algorytmów ewolucyjnych.

Do rozwiązania tego problemu planujemy zastosować sieć MLP, gdzie wejściem sieci
będzie stan rozważanego środowiska, a wyjściem będzie akcja do wyboru przez agenta.
Sieć ta będzie optymalizowana za pomocą algorytmu ewolucyjnego CMA-ES.

\subsection{Planowane eksperymenty numeryczne}

W ramach eksperymentów ocenimy działanie implementacji naszego algorytmu 
neuroewolucji na różnych środowiskach OpenAI gym.

Następnie porównamy jego działanie z dwoma algorytmami RL pod kątem 
końcowej nagrody oraz czasu uczenia algorytmu. Wybrane przez nas 
algorytmy RL to A2C oraz PPO.


\subsection{Biblioteki wybrane do realizacji projektu}

\begin{itemize}
  \item \textbf{\href{https://github.com/openai/gym}{gym}} - liczne środowiska, w tym kontroli klasycznej, 
  do trenowania agentów.
  \item \textbf{\href{https://github.com/CyberAgentAILab/cmaes}{cmaes}} - 
  implementacja algorytmu CMA-ES.
  \item \textbf{\href{https://github.com/DLR-RM/stable-baselines3}{stable-baselines}} -
  biblioteka zawierająca gotowe implementację algorytmów RL.
\end{itemize}

\begin{thebibliography}{9}
\bibitem{scalable_alternative} 
Tim Salimans, Jonathan Ho, Xi Chen, Szymon Sidor, Ilya Sutskever,\\
Evolution Strategies as a Scalable Alternative to Reinforcement Learning,
2017, \href{https://arxiv.org/pdf/1703.03864v1.pdf}{https://arxiv.org/pdf/1703.03864v1.pdf}

\end{thebibliography}
  
\end{document}
